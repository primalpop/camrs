% Use the following line _only_ if you're still using LaTeX 2.09.
%\documentstyle[icml2011,epsf,natbib]{article}
% If you rely on Latex2e packages, like most moden people use this:
\documentclass{article}

% For figures
\usepackage{graphicx} % more modern
%\usepackage{epsfig} % less modern
\usepackage{subfigure} 

% For citations
\usepackage{natbib}

% For algorithms
\usepackage{algorithm}
\usepackage{algorithmic}

% As of 2010, we use the hyperref package to produce hyperlinks in the
% resulting PDF.  If this breaks your system, please commend out the
% following usepackage line and replace \usepackage{icml2011} with
% \usepackage[nohyperref]{icml2011} above.
\usepackage{hyperref}

% Packages hyperref and algorithmic misbehave sometimes.  We can fix
% this with the following command.
\newcommand{\theHalgorithm}{\arabic{algorithm}}

% Employ the following version of the ``usepackage'' statement for
% submitting the draft version of the paper for review.  This will set
% the note in the first column to ``Under review.  Do not distribute.''
\usepackage[accepted]{icml2011} 
% Employ this version of the ``usepackage'' statement after the paper has
% been accepted, when creating the final version.  This will set the
% note in the first column to ``Appearing in''
% \usepackage[accepted]{icml2011}


% The \icmltitle you define below is probably too long as a header.
% Therefore, a short form for the running title is supplied here:
\icmltitlerunning{Context Aware Movie Recommender Systems}

\begin{document} 

\twocolumn[
\icmltitle{CAMR - A Context Aware Movie Recommender System}

% It is OKAY to include author information, even for blind
% submissions: the style file will automatically remove it for you
% unless you've provided the [accepted] option to the icml2011
% package.
\icmlauthor{Primal Pappachan}{primal1@umbc.edu}
 %\icmladdress{Your Fantastic Institute,
%            314159 Pi St., Palo Alto, CA 94306 USA}
\icmlauthor{Arnav Joshi}{arnavj1@umbc.edu}
%\icmladdress{Their Fantastic Institute,
%            27182 Exp St., Toronto, ON M6H 2T1 CANADA}

% You may provide any keywords that you 
% find helpful for describing your paper; these are used to populate 
% the "keywords" metadata in the PDF but will not be shown in the document
\icmlkeywords{recommender systems, machine learning}

\vskip 0.3in
]

\section{Introduction}
The task of recommender systems is to turn data of users and their preferences into predictions of possible future likes and interests \cite{adomavicius2011context}.  The majority of existing approaches to recommender systems focuses on recommending the most relevant items to individual users. However, they do not take into consideration any contextual information, such as time, place and the company of other people (in use cases such as movies). In other words, traditional recommender systems deal with applications having only two types of entities, users and items, and do not put them into a context when providing recommendations. \\

Context information is any information about the situation, circumstances and user state when a user is consuming the content item. Context can be the time of day, weather, social situation, user’s mood etc. However, the relationship between context and user decision making is very complex and difficult to model. 

Context-aware recommender systems help users and their desired content in a reasonable time, by exploiting the pieces of information that describe the situation in which users will consume the items. CAMRS models this approach by providing the recommendation on the basis of a function of user, item, previous ratings, as well as the context information provided by the user.

\subsection{Importance of context} 
The inclusion of the contextual information into the recommendation process presents opportunities for richer and more diverse interactions between the end-users and recommender systems. With the help of user's context, CAMRS will not only incorporate the user's profile of previous movies he has watched, but also include domain-dependent context modeling. Adapting the item choice based on the user's context can help recommend movies that might be given better ratings by the user.

\subsection{Generating Recommendations}
Traditional Collaborative Filtering systems gather item ratings as a form of user feedback for items in a given domain and exploit similarities and differences among profiles of several users in determining how to recommend an item. In CAMRS we hope to gain additional insight on user preferences by taking contextual information into consideration as explicit categories of data, such as the time, location and social situation(with a companion or not). The rating function r can thus be defined as: 

\begin{equation}
r: User * Item * Context = Rating 
\end{equation}
The rating will be on a Likert scale (scale of 1-5) and detection of movie relevance is done on the basis of rating information. 

\section{Related Work}
\cite{melville2002content} incorporated both content based and collaborative filtering methods for recommender systems. Their approach uses a content-based predictor to enhance existing user data, and then provide personalized suggestions through collaborative filtering. Collaborative filtering systems are affected by two problems:
\begin{description}
\item[Sparsity] Due to overwhelming amount of choices, the user-item rating matrix is sparse. Hence finding a sizable amount of similar ratings for the active user is usually low.\item[First-rater problem] An item can be recommended only when it is previously rated by a user.
\end{description}
The drawbacks for Collaborative Filtering (CF) systems were mitigated by exploiting content information of the items already rated. The basic approach uses content-based predictions to convert a sparse user ratings matrix into a full ratings matrix; and then uses CF to provide recommendations. The approach introduces a new model - Content-Boosted Collaborative Filtering (CBCF). The approach was proven to perform better than both pure content-based and collaborative filtering approaches.

\cite{adomavicius2011context} identifies the impact of context on recommender systems, models and predicts user tastes and preferences by incorporating available contextual information into the recommendation process as explicit additional categories of data. The approach also introduces three different algorithmic paradigms – contextual pre-filtering, post-filtering, and modeling – for incorporating contextual information into the recommendation process. The contextual post-filtering paradigm is of significance, which uses a two-fold technique for filtering results:

\begin{description}
\item[Filtering] the recommendations such that only the items which are relevant to the active user are retained.
\item[Adjusting] the ranking of recommendations based on the context filters applied on the filtered results.
\end{description}

Further, they discuss the possibilities of combining several context- aware recommendation techniques into a single unifying approach.

\section{Proposed method}

Our approach comprises the following steps in building the Context-Aware Movie Recommender(CAMR) system

\begin{enumerate}
\item Build a recommendation system based on collaborative filtering
\item Compute the Information gain on contextual attributes per user. Construct a user profile with these attributes in descending order of Information Gain 
\item Filter the recommendations based on contextual attributes from the user profile 
\item Use feature selection methods over the dataset to extract relevant contextual attributes
\item Use K-means clustering to identify the average attributes of instances with high rating
\end{enumerate}

The effectiveness of various steps mentioned above was evaluated using the following methods
\begin{itemize}
\item Compute the total number of ratings given by a specific user 
\item Reserve $2/3rd$ of the total ratings given by the user for training
\item Utilize the rest of the data for testing
\item Compute Precision, Recall and F1-Score metrics on the testing set 
\end{itemize}

This model will take a fraction of the dataset for training, and generate the recommendation (a ranked list of movies sorted in descending order of ratings) for a given user. The recommendations generated by the previous step as well as by filtering based on contextual attributes are compared using various evaluation metrics.

\begin{figure}[H]
\centering
\includegraphics[width=3.5in]{archdiagram.jpg}
\caption{Contextual post-filtering for recommendation systems}
\label{archdiag}
\end{figure}

Figure ~\ref{archdiag} shows the overall architecture of the recommendation system. The \textit{Data Model} represents the pre-processed dataset where the dataset is read from a csv file and stored into two separate data stores corresponding to user and movie attributes respectively. Based on the similarity metric, other users who have given similar ratings to the user under consideration were found out.     The two similarity metrics used were

\begin{itemize}
\item Euclidean distance 
\item Pearson correlation
\end{itemize}

The results from previous step were used in Collaborative Filtering which generates a recommendation of the movies for a given user by comparing with other users with similar tastes. These recommendations are then passed through the contextual filtering step, where the results are narrowed down based on contextual attributes(user-specific or general) computed using different methods (specified in following sections). Further, we compute a contextualized result in the form of a ranked list of movie recommendations. The block \textit{U} represents the active user under consideration for the recommender system and \textit{C} represents the context for the contextual-filtering step. The correlation metric helps refine the context attribute for the filtering approach. The following sections describe in detail each of these steps.

\subsection{Dataset}
We used the LDOS - CoMoDa dataset \cite{kovsir2011database}, made available by Dr. Andrej Kosir from the University of Ljubljana, Slovenia. The LDOS - CoMoDa dataset is a context-rich movie recommender dataset, which contains ratings for movies along with contextual information describing the situations in which the movies were watched. 
It contains 30 variables among which are 12 contextual variables. The attributes can be classified as:
\begin{enumerate}
\item \emph{User Attributes}: ID, Sex, Age, City, Country
\item \emph{Movie Metadata}: rating, director, language, country, genre(s), actor(s), budget
\item \emph{Context attributes}: time,weather, daytype, season, location, social, end emotion, dominant emotion, mood, physical
\end{enumerate}
The value for each non-numerical discrete attribute is represented with a distinct number. The training instances for the recommendation system consists of the user, movie and context attributes for a single user and the corresponding rating associated with it. A single user can have multiple ratings, indicating he has rated more than one movie. The dataset consists of 2296 instances (ratings), with 121 unique users. 

\subsection{Cleaning the data}

The LDOS-CoMoDa dataset was converted into csv format to make it easy for parsing. The dataset is read and stored in pandas dataframe object \footnote{http://pandas.pydata.org/}, replacing all the missing values(-1) using the mean of the available values. We maintain two separate dictionaries for user attributes (indexed by the user ID) and movie attributes (indexed by the movie ID). The movie attributes are mapped to individual users by the movie ID, as part of the values associated with each user instance. The structure of these dictionaries are given below.

users = 
	{userid: 
		{itemid:  
			rating, 
			age,
			sex,
			city,
			country, 
			c1,
			.
			.
			c12}}

items = 
	{itemid: 
		{director, 
		a1..a3,
		g1..g3, 
		budget, 
		language, 
		country}}
$c_{i}$ denote contextual attributes.

\subsection{Collaborative Filtering}
In Collaborative Filtering, the recommendations are generated based on a user and her previous movie ratings. The first step in this process is to compute the similarity score to estimate users that are most similar to a given user. Higher values of similarity score mean that the pair of users under consideration are similar. After finding out the users with similar tastes, we sort them in descending order to obtain a ranked list of similar users. To generate a recommendation, the items were ranked the users by a weighted score. The movie ratings from similar users were multiplied with the similarity score to obtain the weighted score for each unseen movie. These were returned in descending order of the weighted score as a recommendation. \cite{segaran2008programming}  

\subsection{Building the User Profile}

A User Profile consists of contextual attributes which are used for contextual filtering of recommendations. The following methods were used for finding out relevant attributes from the set of context features.

\subsubsection{Feature Selection}

We experimented on alternative techniques on characterizing profiles based on user similarities in terms of the contextual attributes. The first method was used to find out the existence of important contextual features, to boost the performance of CAMRS. The forest of trees approach was used to evaluate the importance of features in the dataset. Tree-based estimators from the sklearn.tree module and forest of trees in the sklearn.ensemble module were used for the purpose\cite{scikit-learn}. Univariate chi-square feature selection was also tried for the purpose. However, chi-square selection did not return useful results.

\subsubsection{Information Gain}

Information gain helps bias the decision making process by considering features that are relevant\footnote{http://en.wikipedia.org/wiki/Information_gain_ratio}. Information gain is particularly effective, since the attributes in LDOS-CoMoDa are primarily nominal and take up a finite (small) set of values. The ratings given by each user were considered in isolation for computing the information gain of each attribute. Compared to the first method, this method a tailored set of attributes specific to each user. We built a user profile for every user in the dataset, based on the context attributes with the highest information gain. This calculation was done on the basis of the ratings given for a movie watched by the user, under the context features described for the same.

\subsubsection{Clustering}

 We used the K-means clustering algorithm to find out the clusters of user profiles. The algorithm was based on finding similarity between instances based on the Euclidean distance between contextual attributes that are associated with a profile. The clustering algorithm was evaluated for five clusters, based on the five possible ratings. Each cluster was observed to show distinct features which may have influenced the user ratings.

\subsection{Contextual filtering}
In this module, we refine the results provided by the Collaborative Filtering system, based on the filtered context features from the previous step. The user enters a set of attributes describing his context which would act as the filters. Contextual Filter procedure takes the following as input.

\begin{itemize}
\item UDB: User - Movie ratings data dictionary
\item User: Specific user under consideration 
\item Filters: User entered Contextual Attributes	
\item Recommendations: Generated by Collaborative Filtering
\item Profiles: User Profile built based on Information gain of Contextual attributes
\end{itemize}

\begin{algorithm}[tb]
   \caption{Contextual Filter}
   \label{Contextual Filtering of Collaborative Filtering results}
\begin{algorithmic}
   \STATE {\bfseries Input:} $UDB$, $User$, $Filters$, $Recommendations$, $Profiles$
   \FOR{$rating, movie$ {\bfseries in} $Recommendations$}
   \IF{$rating$ >= {\bfseries OPTIMUM}} 
   \FOR{$context, value$ {\bfseries in} $filters$}
   \IF{$context$ {\bfseries in} $profiles_{user}$}
   \STATE $avg$ = {\bfseries findContext}($movie, context, udb$)
   \IF{$avg == value$}
   \STATE $filteredRecommendations$ += $((rating, movie))$
   \ENDIF   
   \ENDIF
   \ENDFOR
   \ENDIF
   \ENDFOR
\end{algorithmic}
\end{algorithm}

The $findContext$ procedure was implemented in two different ways. The first method($findAvgContext$) finds the average context of a specific movie with high rating, over all users. The second method ($findMaxContext$) finds the maximum repeating context for a high rating movie, over all users. Once the context has been computed by one of the above two methods, if this context is equal to the context entered by the user, the specific movie is added to the filtered recommendations.

%\section{Intuition}

\section{Experiments and Results}
Table~\ref{ldos} demonstrates the statistics of the LDOS - CoMoDa dataset. These demographics help identify the correlation not only between the contextual attributes and the movie's ratings, but also includes user profiles. For example, it was observed that when the dominant mood was "Happy", the overall ratings, regardless of the other attributes, were high (ranging from 4 to 5). This sets up a correlation, of the contextual attribute modeling the mood of the user, to be a significant parameter for recommending a movie with higher rating. (see Appendix)

\begin{table} [H]
	\caption{LDOS - CoMoDa dataset statistics}
	\label{ldos}
	\vskip 0.15in      
	\begin{center}
	\begin{small}
	\begin{sc}
    \begin{tabular}{lcccr}
	\hline
	\abovespace\belowspace
	LDOS-CoMoDa statistics\\
    \hline
    \abovespace
    Number of users            & 121 \\
    Number of items            & 1233  \\
    Number of ratings          & 2294  \\
    Average age                & 28  \\
    Number of countries        & 6  \\
    Number of cities           & 18  \\
    Max ratings of single user & 220  \\
    \belowspace
	Min ratings of single user & 1  \\
	\hline
	\end{tabular}
	\end{sc}
	\end{small}
	\end{center}
	\vskip -0.1in
\end{table}

\subsection{Evaluation Metrics}
An effective method to evaluate a recommender's suggestions, is to grade the quality of its estimated preference values. This involves measuring how closely the estimated preferences match the actual preferences of the user. In order to evaluate the accuracy of CAMRS , we use a precision-recall measure where the predicted movie recommendations, and their respective ratings, are compared with the movies in test set. \\

The precision metric finds the proportion of relevant movies from the list of (movie, rating) recommendation pairs. Here a relevant movie   is the one which is present in the test set and has a rating similar to what is predicted by CAMRS.
\begin{equation}
precision (per user) = \frac{\text{No.\,of\,rel.\,movies}} {\text{No.\,of\,reco'd\, movies}}
\end{equation} 
The recall measures the ratio of good movies that were recommended from the total list of all possible good movies. Here a good movie is the one which was given a high rating by the user under consideration.
\begin{equation}
\begin{tiny}
recall (per user) = \frac{\text{Good\,movies\,reco'd }}{\text{Total\,no.\,of\,good\,movies}}
\end{tiny}
\end{equation}
 The F1 score is another metric to evaluate accuracy.  It considers both precision and recall measures of a test to compute the score. In recommendation systems, it is considered a single value obtained combining both the precision and recall measures and indicates an overall utility of the recommendation list.\footnote{http://aimotion.blogspot.com/2011/05/evaluating-recommender-systems.html}
\begin{equation}
\begin{tiny}
F1 score = 2 * \frac{precision * recall}{precision + recall}
\end{tiny}
\end{equation}

 We calculated the precision, recall and the F1 score metrics for the following 
\begin{enumerate} 
\item Results based on the Collaborative Filtering
\item Results filtered by contextual features selected with high Information Gain. Selection of value for the context filter was implemented in 2 ways
\begin{enumerate}
\item By finding average context 
\item By finding maximum context
\end{enumerate}
\end{enumerate}

\subsection{Precision Recall with collaborative filtering}
 As mentioned above, the LDOS-CoMoDa dataset was split into training and test sets based on the user under consideration. The training data was initially passed through the collaborative filtering step. Each (movie, rating) pair in the list of generated recommendations is then compared with the movies and subsequent ratings in the test dataset. The movies with a $\pm$1 difference in rating are considered as a match. Using this metric, the precision for each user in the test dataset was calculated over the list of all recommendations, and was found to be in the range of 0.02 and 0.05. For the recall measure, the ratio of the list of matched ratings was compared with the total list of "good" movies. A "good" movie was considered to be one having a rating of $\ge$ 3.5. Further, the F1 score was calculated based on the observed precision and recall values per user, and was low due to lower values of precision. The observed values for recall were significant compared to the precision measure. Fig~\ref{precision}, Fig~\ref{recall} and Fig~\ref{f1score} demonstrate the comparison of precision, recall and F1 score for different users in the test dataset obtained by Collaborative Filtering and by calculating the maximum value for each contextual attribute via Information Gain filtering
 %For each user in the test dataset, one-third of the movies already rated are removed. 
 
\subsection{Precision Recall after Contextual Filtering}

The contextual attributes identified in User Profile building step were used in two ways to estimate the value based on which the results from Collaborative Filtering recommendations should be filtered. 

\subsubsection{Averaging Context}

The values of contextual filters were averaged over all instances of the recommended movie with high rating. If this value was found to be identical to the one entered by user, then the movie under consideration is added to list of recommendations otherwise it is ignored. The intuition behind this technique was that if when movie was rated high, it had the average value for the filtering contextual attribute is equivalent to that currently entered by the user, then there is a higher chance of being rated high.\\

The precision-recall and F1 scores for the recommendations, generated through this method, were found to be lower than those with the simple recommendation process. Fig~\ref{precrecallavg} shows the sample distribution over the test dataset. A possible explanation could be that averaging over categorical context values might result in sporadic values for context which are not actually representing the average value of context when movie was rated high. It was observed that in cases where contextual attributes had large differences in Information gain, it resulted in slightly improved performance. We also suspect that context values are skewed as most of the movies had only 1 or 2 of the possible values for context out of 4 or 5 possibilities.\\

\begin{figure}[H]
\includegraphics[height=2.5in, width=3.5in]{PrecRecallIG.png}
\caption{Precision Recall Metrics for recommendations filtered on Average values of contextual attributes}
\label{precrecallavg}
\end{figure}

\subsubsection{Maximum Repeated Context}

Since averaging over values of the contextual attributes was found not to improve the performance we tried filtering based on the maximum repeated value of the attribute. For this the most frequent value of contextual filters were found out over all instances of the recommended movie with high rating. If this value was found to be identical to the one entered by user, then the movie under consideration is added to list of recommendations, otherwise ignored. The intuition behind this technique was that if the maximum repeating value of the context when movie was rated high, for the filtering contextual attribute, is equivalent to that current value entered by the user, then there is a higher chance of the movie being rated high.\\

The precision and F1 scores for the recommendations, generated through this method, were found to be higher than those with the simple recommendation process. In many of the cases the precision was found to have improved 3 or 4 times compared to that of Collaborative Filtering. This is in alignment with our expectation that context plays an important role in rating given by the user to the movie. Fig~\ref{precision}, Fig~\ref{recall} and Fig~\ref{f1score} demonstrate the comparison of precision, recall and F1 score for different users in the test dataset obtained by Collaborative Filtering and by calculating the maximum repeated content for each contextual attribute via Information Gain filtering.

\begin{figure}[H]
\includegraphics[height=2.5in, width=3.5in]{Precision.png}
\caption{Comparison of precision measures for collaborative filtering and maximum repeated values in contextual attributes}
\label{precision}
\end{figure}

\begin{figure}[H]
\includegraphics[height=2.5in, width=3.5in]{Recall.png}
\caption{Comparison of recall measures}
\label{recall}
\end{figure}

\begin{figure}[H]
\includegraphics[height=2.5in, width=3.5in]{F1Score.png}
\caption{Comparison of F1 scores}
\label{f1score}
\end{figure}

\begin{table}[t]
\label{kmeans}
%\vskip 0.15in
\begin{tiny}
\begin{tabular}{p{1.4cm} p{0.9cm} p{0.9cm} p{0.9cm} p{0.9cm} p{0.9cm} p{0.9cm}}
\hline
\abovespace\belowspace
Attribute & Full Data (2296) & Rating 1(267) & Rating 2(322) & Rating 5(721) & Rating 3(596) &  Rating 4(390) \\
\hline
\abovespace
age         &  28.418  & 31.7963  & 25.6773 & 29.6492  & 28.3657  & 26.1719 \\
sex         &    Male  & Female   & Male    & Male     & Female   & Male    \\
country     &      UK  & UK       & Denmark & UK       & UK       & Denmark \\
time        & Evening  & Night    & Evening & Evening  & Evening  & Night   \\
daytype     & Weekend  & Weekend  & Weekend & Working  & Working  & Working \\
season      &  Winter  & Autumn   & Autumn  & Autumn   & Winter   & Winter  \\
location    &    Home  & Home     & Home    & Home     & Home     & Home       \\
weather     &   Sunny  & Rainy    & Sunny   & Sunny    & Cloudy   & Cloudy     \\
social      &   Alone  & Alone    & Alone   & Partner  & Partner   & Partner    \\
endEmo      &   Happy  & Neutral  & Neutral & Happy    & Happy  & Happy    \\
dominantEmo & Neutral  & Disgusted & Neutral & Neutral & Happy    & Happy    \\
mood        & Positive & Neutral  & Neutral & Positive & Neutral  & Positive   \\
physical    & Healthy  & Ill  & Ill & Healthy  & Healthy  & Healthy    \\
decision    & Decision & Decision & Given   & Decision & Decision & Given      \\
\belowspace
interaction & First    & First    & First   & First    & First    & First      \\
\hline
\end{tabular}
\end{tiny}
\caption{K-Means clustering on the LDOS-CoMoDa dataset}
\end{table}

\subsection{Clustering based on User profiles}
 Table 2 gives a summary of the clustering experiment, and gives the average of the values for the contextual features across the entire dataset. Further, it also gives the average value for attributes, characterizing the choice of a particular rating by a user. It was observed that users opting for their dominant emotion to be Happy, rated the movies with a higher rating (4 or 5). Also, users gave higher ratings when they watched movies on a working day, with a partner, during the winter season and during cloudy weather. The age group which gave lower ratings (1) were mostly females in their '30s. The K-means clustering thus gave comprehensive statistics on the distribution of data.

\section{Conclusions}
 The CAMR system demonstrates the usage of contextual information as a relevant set of features, when providing movie recommendations. The system not only incorporates user tastes with similar profiles and ratings, but also filters the recommendations based on the current context of the active user. The recommendations are improved by understanding the correlation metrics between features that affect the outcome of the movie ratings, and thereby altering the filtering approach. We analyzed the importance of context on the recommendation outcome using Information Gain of attributes. From the precision-recall values of Collaborative Filtering and Contextual Filtering we identified the importance of context for movie selection. We also hope that in a system where contextual information can be collected automatically rather than through manual entry, the problem of contextual values being defaulted can be avoided. As of now we have reasons to believe that incorporating a rich set of contextual features from the dataset looks promising, and with adequate (complete) data, we can provide significant value addition to a movie recommender system.

\section{Future Work}

\begin{enumerate}
\item Sentiment Analysis of text from social networks(twitter, facebook) and message, call log can be used to identify the contextual attributes of the user before and after watching the movie
\item By identifying groups of users who are related, values contextual attributes can be inferred for users from the same group
\item Group recommendations filtered on the contextual values for the entire group
\end{enumerate}

\begin{figure}[H]
\includegraphics[height=2.5in, width=3.5in]{numberrecos.png}
\caption{Effect of contextual postfiltering on the number of generated recommendations}
\label{numberrecos}
\end{figure}

\newpage
\section{Appendix}
\subsection{Dataset Analysis}

\begin{figure}[H]
\centering
\includegraphics[height=2.5in, width=3.5in]{ratingsperuser.png}
\caption{Ratings per user}
\label{ruser}
\end{figure}

\begin{figure}[H]
\centering
\includegraphics[height=2.5in, width=3.5in]{ratungspermovie.png}
\caption{Ratings per item}
\label{ritem}
\end{figure}

\bibliography{pr_ref}
\bibliographystyle{icml2011}
\end{document}