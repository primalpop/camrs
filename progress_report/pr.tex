% Use the following line _only_ if you're still using LaTeX 2.09.
%\documentstyle[icml2011,epsf,natbib]{article}
% If you rely on Latex2e packages, like most moden people use this:
\documentclass{article}

% For figures
\usepackage{graphicx} % more modern
%\usepackage{epsfig} % less modern
\usepackage{subfigure} 

% For citations
\usepackage{natbib}

% For algorithms
\usepackage{algorithm}
\usepackage{algorithmic}

% As of 2010, we use the hyperref package to produce hyperlinks in the
% resulting PDF.  If this breaks your system, please commend out the
% following usepackage line and replace \usepackage{icml2011} with
% \usepackage[nohyperref]{icml2011} above.
\usepackage{hyperref}

% Packages hyperref and algorithmic misbehave sometimes.  We can fix
% this with the following command.
\newcommand{\theHalgorithm}{\arabic{algorithm}}

% Employ the following version of the ``usepackage'' statement for
% submitting the draft version of the paper for review.  This will set
% the note in the first column to ``Under review.  Do not distribute.''
\usepackage[accepted]{icml2011} 
% Employ this version of the ``usepackage'' statement after the paper has
% been accepted, when creating the final version.  This will set the
% note in the first column to ``Appearing in''
% \usepackage[accepted]{icml2011}


% The \icmltitle you define below is probably too long as a header.
% Therefore, a short form for the running title is supplied here:
\icmltitlerunning{Context Aware Movie Recommender Systems}

\begin{document} 

\twocolumn[
\icmltitle{CAMR - Context Aware Movie Recommender Systems}

% It is OKAY to include author information, even for blind
% submissions: the style file will automatically remove it for you
% unless you've provided the [accepted] option to the icml2011
% package.
\icmlauthor{Primal Pappachan}{primal1@umbc.edu}
 %\icmladdress{Your Fantastic Institute,
%            314159 Pi St., Palo Alto, CA 94306 USA}
\icmlauthor{Arnav Joshi}{arnavj1@umbc.edu}
%\icmladdress{Their Fantastic Institute,
%            27182 Exp St., Toronto, ON M6H 2T1 CANADA}

% You may provide any keywords that you 
% find helpful for describing your paper; these are used to populate 
% the "keywords" metadata in the PDF but will not be shown in the document
\icmlkeywords{recommender systems, machine learning}

\vskip 0.3in
]

\begin{abstract} 
We propose a context aware movie recommendation system which computes movie suggestions based on models and predictions 
from user preferences, and incorporate available contextual information into the recommendation process. 
\end{abstract} 

\section{Introduction}
The task of recommender systems is to turn data on users and their preferences into predictions of possible future likes and interests[4]. A rating function in a recommendation system is one which tries to estimate the rating for the new item (here, the movie choice) based on the user’s profile i.e. previous preferences. In CAMRS we hope to gain additional insight on user preferences by taking contextual information into consideration as explicit categories of data, such as the time, location and social situation(with a companion or not). The rating function r can thus be defined as: 
r: User x Item x Context = Rating [2]
The recommendation given by CAMRS is on the basis of a function of user, item and previous ratings.  The recommendation will be on a Likert scale (scale of 5-10 or a Boolean variable “Like/Dislike”). Ratings are measured on a 5-10 scale and detection of movie relevance done on the basis of rating information. Following are examples of some of the contextual attributes and their values in the dataset.
\section{Related Work}

\section{Proposed method}

\section{Experiments}

\section{Conclusions}

\bibliography{pr_ref}
\bibliographystyle{icml2011}

\end{document} 
